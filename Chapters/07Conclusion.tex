
%Give a concise summary of your research and finding here, and include a short summary of any future work as well.

The objective of this research was to make a digital world for benefiting different sectors specifically marine industry. 

\noindent In the following, we revisit the raised research questions, indicating associated key findings:\\

\noindent $RQ_1-$ What are the benefits of scanning an object and making point cloud?  \\

\noindent Point clouds can precisely calculate distances, volumes, angles, and other geometric features. They provide suitable precision, down to the millimeter, which eliminates mapping inaccuracies. Point-cloud models can properly depict the features of almost any 3D object if the resolution and point density are sufficient. The denser the points, the more realistic the representation, which clearly distinguishes minor features and texture details.Point cloud models are the quickest to produce, assuming you have access to 3D scanning technology. Point-cloud data has numerous applications in simulation. It can be used to represent solid things in a Finite Element Analysis framework, allowing engineers and scientists to simulate objects under stress, deformation, and so on. There are two approaches for point cloud data capturing, one is a human operator based capturing, this method is simplest and less-expensive way for data capturing, you can take your smartphone and take a 360 degree film around the object to cover all angles that you need for your model then you can process the video in a software to make point cloud, finally it is ready to add mesh or texture for 3D modeling. (This method is used in this thesis). And the other approach is ground-based or aerial-based robot capturing, in this method we have a broader view in drone-based robot and advantage of constant speed capturing with less noisy data for both ground-based or aerial-based robot. \\


\noindent $RQ_2-$ What are the advantages of using captured point cloud for 3D reconstruction purposes with application in construction industry? \\
After scanning the building or location that you want to do reconstruction, we import the point cloud data to Autodesk Recap Pro software to use the feature that called "scan to mesh". Then you can select all or portion of the captured point cloud in Recap Pro to make a low-medium or high quality mesh. Then your data is ready for architectures or designers to rebuild parts of structure in CAD software. \\

\noindent $RQ_3-$  How to benefit from captured point cloud for more efficient ship maintenance? (This question addressed in literature review)\\
After scanning of the intended rooms or components in the ship for making a dense point cloud, we can work on this digital representative of the physical ship.  
The next step is pre-processing of dense point cloud for better handling of data by implementing semantic segmentation. Segmentation is helpful for suitable differentiation of the installed components. In the next step of the model rebuilding, we need to extract the geometrical data from point cloud, this stage is vital in re-engineering of retrofitting process. \\

\noindent $RQ_4-$ How to investigate the simulation of a robot in a 3D virtual world which is made from point clouds for safer marine operations?\\
We need a high fidelity virtual environment with real-world physics for this rescue robot to have an interactive and realistic training scenarios. This environment could be a point cloud or a textured model. The introduction of virtual environments has transformed the area of robotics, notably in terms of simulating a virtual blind robot (without a vision sensor and just equipped with a laser scanner) in a tailored interior environment. This strategy, which takes into account obstacle avoidance and goal tracking requirements, provides several benefits, particularly when the end goal is to ease rescue operations. This approach's first advantage is familiarity. By simulating in our own environment, we have complete control over the location's known layout and features. This enables for easier design adaptability to various test scenarios.Privacy is another key advantage. Using our own simulated settings provides privacy, which is a problem when using other people's houses for the same purpose.
Customizability is an additional benefit of simulation in our built environment. Similar to familiarity, we can modify the lighting, furniture, and items in the setting to achieve the desired simulation goals.
Relevance is another advantage of simulation in our organized space. The settings in our own environment are more likely to be relevant to the jobs we want the robot to perform. For example, if we're building a robot to help with household chores, it makes logical to replicate those jobs in the same surroundings.Operational realism is another benefit. This refers to an exact portrayal of the robot's actual activities, which includes the robot's kinetic and dynamic properties, as well as its interactions with the environment. The primary goal of operational realism is to create a robot program that performs ideally in real-world circumstances.
This form of simulation also has the advantage of generating data. Data-driven algorithms require a big amount of high-quality data to perform properly. Synthetic data creation is gaining popularity due to its speed and automatic annotation.
Cost-effectiveness is a key aspect of virtual simulation. Testing robots in virtual environments can significantly reduce the costs associated with physical prototypes. There is no need to construct real-world testing facilities or repair physical damage to the robot or its surroundings, which can be costly.Another important element is to provide a safe experimental environment. Virtual environments provide a safe setting in which robots can be tested without endangering real-world property or the robot itself. This is particularly important in the early stages of development, when faults and errors are most likely to arise.Another option for simulation is to train artificial intelligence systems. Virtual environments provide a wide variety of scenarios for training the artificial intelligence systems that run the robots. Simulation-generated datasets are diverse, rich, and well-controlled, making them useful for machine learning methods, particularly reinforcement learning.
Preparing for Unexpected Scenarios can be viewed as an opportunity in simulation. Simulations can subject robots to unique, unexpected environments in order to evaluate their behavior, guaranteeing that the robot can handle a wide range of probable real-world circumstances safely and successfully. Finally, R and D is an important benefit in simulation practice. Virtual simulations allow researchers to test and enhance new algorithms and techniques in a controlled and repeatable environment. \\

\noindent In my research due to lack of accessibility to land-based or aerial robot and also long waiting  time for access to vessel and robots,  I used human-operator-based method for scanning. 